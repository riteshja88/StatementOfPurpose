%%%%%%%%%%%%%%%%%%%%%%%%%%%%%%%%%%%%%%%%%%%%%%%%%%%%%%%%%%%%%%%%%%%%%
%% Title: SOP Ritesh Agarwal
%% Author: Ritesh Agarwal / riteshja88@gmail.com
%% Created: 2017-04-03
%%%%%%%%%%%%%%%%%%%%%%%%%%%%%%%%%%%%%%%%%%%%%%%%%%%%%%%%%%%%%%%%%%%%%

%%%%%%%%%%%%%%%%%%%%%%%%%%%%%%%%%%%%%%%%%%%%%%%%%%%%%%%%%%%%%%%%%%%%%
%%
%%
%% How to Compile:
%%     $ xelatex main.tex
%%
%%%%%%%%%%%%%%%%%%%%%%%%%%%%%%%%%%%%%%%%%%%%%%%%%%%%%%%%%%%%%%%%%%%%%

\documentclass[letterpaper]{article}
\usepackage[letterpaper,margin=1.75in,noheadfoot]{geometry}
\usepackage{fontspec, color, enumerate, sectsty}
\usepackage[normalem]{ulem}

%%%%%%%%%%%%%%%%%%%%%%%%%%%%%%%%%%%%%%%%%%%%%%%%%%%%%%%%%%%%%%%%%%%%%
%                      YOUR INFORMATION
%
%      PLEASE EDIT THE FOLLOWING LINES ACCORDINGLY!!
%%%%%%%%%%%%%%%%%%%%%%%%%%%%%%%%%%%%%%%%%%%%%%%%%%%%%%%%%%%%%%%%%%%%%
\newcommand{\soptitle}{Statement of Purpose}
\newcommand{\yourname}{Ritesh Agarwal}
\newcommand{\youremail}{riteshja88@gmail.com}

%% FONTS SETUP
\defaultfontfeatures{Mapping=tex-text}
\setromanfont[Scale=1.0]{Helvetica Neue Light}
\setmonofont[Scale=0.8]{Helvetica Neue Light}
\setsansfont[Scale=0.9]{Helvetica Neue Light}
\newcommand{\amper}{{\fontspec[Scale=.95]{Helvetica}\selectfont\itshape\&~{}}}
\usepackage[bookmarks, colorlinks, breaklinks,
pdftitle={\yourname - \soptitle},pdfauthor={\yourname}, unicode]{hyperref}
\hypersetup{linkcolor=magneta,citecolor=magenta,filecolor=magenta,urlcolor=[named]{WildStrawberry}}

%%%%%%%%%%%%%%%%%%%%%%%%%%%%%%%%%%%%%%%%%%%%%%%%%%%%%%%%%%%%%%%%%%%%%
%                      Title and Author Name
%%%%%%%%%%%%%%%%%%%%%%%%%%%%%%%%%%%%%%%%%%%%%%%%%%%%%%%%%%%%%%%%%%%%%
\begin{document}
\begin{center}{\huge \scshape \soptitle}\end{center}
\begin{center}\vspace{0.2em} {\Large \yourname\\}
  {\youremail}\end{center}

%%%%%%%%%%%%%%%%%%%%%%%%%%%%%%%%%%%%%%%%%%%%%%%%%%%%%%%%%%%%%%%%%%%%%
%                      SOP Body
% NOTE: Use \amper instead of \&
%%%%%%%%%%%%%%%%%%%%%%%%%%%%%%%%%%%%%%%%%%%%%%%%%%%%%%%%%%%%%%%%%%%%%
\section*{Introduction}

In 1996, computers were new and relatively uncommon in India. 
I was one of those fortunate students to be introduced to it then. 
Since then I have always been inclined towards  them as they  
fascinated me with the assortment  of tasks  that  they  could  perform  with ease.
The passion augmented when I was formally introduced to programming languages and saw 
what few simple and precise commands would do with accuracy.
This enthusiasm has always imbued me to go further and never stop in my pursuit for 
excellence in computer science.

I have completed my formal education in India in the field of Computer Engineering and Masters in Computer Science 
from University of Illinois, Chicago. I have also worked about 2 years before my Masters and 2 years after in 
related areas.
This unique experience has given me insights of the Computer Science field and I believe revisiting the 
education system with the insights will give me a better ability to shape my career by making sure 
I will have he ability absorb the knowledge better,
now that I have practical understanding of the concepts and better insights of 
how things actually happen in a generic computer system.

\section*{Research and Work Experience}
\paragraph{Job \#1 : Mobility NFV Virtualization for Carrier Grade OS}
My most recent work has been at Cisco where I was a part of a Cisco spin-in.
The spin-in basically had a goal to have a Mobility Gateway stack run on Generic
x86 CPU based blades rather than a special purpose Hardware platform. The ability 
to repurpose the blades for other productive use, and the advancements in Virtualization
technology calls for this kind of disruption. I have got a chance to get my hands dirty 
with installing blades, switches, routers on the rack to installing and configuring a gateway
in the VM thats running on the blade and define connectivity as per the need. Most of my work
there allowed me to understand how an OS comes up starting from the Bootloader stage. I also got
a chance to understand the linux process subsystem and thread subsystem better. I got a chance 
to understand how gdb works internally, how packets traverse through various parts of system when 
they enter the port until they leave a port or are consumed. A project where I had to convert a 
32 bit to a 64 bit also gave me an idea of how cross compilers are built and how an individual
component can be cross compiled in any system. A project I worked on also helped me understand
how to scale up an application with more number of CPU,Memory and Network resources(NICs).

\paragraph{Project \#2: IPC Subystem Cleanups in Ethos Kernel}
This was my Master's project at University of Illinois, Chicago under the guidance of 
Prof. Jon Solworth. As PhD. students were prototyping their research work on this
special purpose paravirtualized OS that runs in Xen, they did not care to make sure
that any asynchronous system call posted in the kernel on behalf of a process also needed to be 
cleaned up when the process exited or crashed. That is where the need for this 
project came in. This project involved intensive code browsing and call graphing
the targeted system calls and making sure that we do the best to clean up, release
resources held and undo any partially completed work in the kernel on behalf of the
process. The task becomes trick with processes forking a child.
Once this was done, the next task was to run the test suite written in Go language
and make sure, we did not break something that worked before our changes. I also contributed 
a few test cases.


\paragraph{Job \#3: My First Job }
After completing my undergraduate degree, I joined CISCO as a Software Engineer.
Here I worked in the development and support of GGSN network node in 3G Packet Domain. 
As a part of my work, I was able to extensively debug huge busybox type binaries, 
find out the root cause of a crash from a core-file, and understand code with no documentation 
at all about the code written in C language. 
To do this I had to understand the topologies of Customer network, not only this 
but also different nodes from different vendors were communicating with many non-standard 
Information elements in Messages which lead to complex scenarios making it even difficult to analyze. 
I analyzed unformatted system logs to find the root cause of a problem. 
The extent of debugging went analysing linked lists in full cores. 
I came up with an Automated way of testing by extending the Python Unit Testing Framework and forming 
a testing framework over it to suit the needs of the product testing. 
Doing this was possible since the majority of test tools that we used before for manual testing was written in Python. 
With the framework testing for regression testin became much easier and avoided repetitive work. 
I also got a chance to do extensive programming in Python. 
With good knowledge of C-language for writing programs where performance matters and Python for writing programs, 
where we are least concerned about performance but are more interested in getting things quickly, 
I feel I can work efficiently anywhere. 



\paragraph{Project \#4: A Reconfigurable Virtual Storage Device}
This was my final year undegraduate project. This project was guided by Prof. Jaiswal from PICT and Ratnadeep Joshi from Marvell Semiconductors(Toshiba).
We implemented and evaluated a virtual reconfigurable storage device consisting of a combination of disks of different 
characteristics that is reconfigurable w.r.t to performance and space. The major focus was on the automatic performance reconfiguration 
of the storage device wherein the access pattern of the disk is analyzed to characterize the groups of disk blocks. 
Our solution moved block groups which are more frequently and randomly accessed into Solid State Drive and move back less frequently block 
groups from Solid State Drive to Hard Drive, while at the same reduced SSD wear by demoting excessive written block groups out of SSD. 
We took care of things like what if we are relocating/swapping blocks from one device and system halts or crashes we should not corrupt the data.
Since our device driver was a stacking device driver over many drives, we ensured that requests given to devices have been completed. 
For e.g the scheduler in the lower device driver might cause a request to stall for a very long time. 
This project was inspired from Technical papers in FAST, IEEE and ACM. The main intention of undertaking this project was more 
of learning while implementing the project.



\section*{Education}
My inclination towards Computers forced me to take a different path towards becoming a computer engineer. 
Instead of going to junior college, I enrolled myself to a Diploma in Computer Engineering course, 
where we were taught Computer Science concepts right from the second year of the three year course proved my 
excellence to the professors right in the first year by securing 97\% marks in Computer Fundamentals in 
the First Year. Moreover, I topped the college all 3 years. My good academics in Diploma helped me secure admission 
in the coveted Pune Institute of Computer Technology (PICT) directly into the Second Year. 
A First class with Distinction in my Bachelor's Degree is a credible proof of my academic performance. 
Curriculum was structured in a way that while stressing over basic concepts and theoretical ideas, 
importance was given to practical implementation wherever possible.  
I learned tremendously during my coursework as well as lab practice, which gave me a clear 
picture of my interests and forte. The syllabi there ranged from subjects like Data Structures, 
Object Oriented Programming and Discrete Structures to Operating Systems, Networking, Computer Architecture 
and Organization and Microprocessor Systems. I learned that experiments, novel ideas, implementation plans 
and reliable tests were the key basis of modern day development. As a part of my education, I have gained knowledge about various concepts in OS, 
Microprocessor, Data Structures, and Linux Kernel and have also implemented a stacking storage device driver. 
I have also written many useful application programs in C, C++, Assembly Language, Visual Basic with Oracle and some small applications in Java and BASIC.
After my undegraduate degree an two years of work experience I got a chance to complete my advanced degree at University of Illinois, Chicago with an
average GPA of 3.85.


\section*{Concluding Remarks}
To summarize my work experience, I have about 4 years of industry work experience, 4
years of intesive academic work experience through various projects undertaken. 
I hold a Bachelors and Masters degree from reputed schools in India and US with really good GPAs.
I wish to now shape my profile with another education opportunity. Now, that I have a better
understandig of what I want to be, it makes much more sense to me to consider taking this step.

After careful considerations, I have decided to go in for another graduate study that calls for personal commitment to 
fulfilling the crafts of independent research and dedicated involvement.
On in-depth study of your website, I have found the graduate program suits my interests.
I have chosen to pursue graduate study at the University because by working under the guidance of the 
distinguished advisors and professors, I am confident of making original contributions in the field of Computer Science. 
I am fully aware that your curriculum requires that I beckon all my resources and 
I maintain that I have the necessary commitment, intelligence and stamina to look forward to it all. 
Thus, I am forwarding my application for admission into your esteemed University and request you to grant me 
an opportunity to enrich my knowledge by interacting with great scholars of your institution. 
I would be grateful if you can consider my candidature favorably for admission with financial support. 
I look forward to the discovery and the challenge of my chosen career.

\end{document}
